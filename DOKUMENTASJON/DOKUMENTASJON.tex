\documentclass[a4paper,9 pt]{article}

\usepackage[utf8]{inputenc}
%\usepackage[T1]{fontenc}
%\usepackage{babel}
\usepackage{graphicx}

\usepackage[font=small,format=plain,labelfont=bf,up,textfont=it,up]{caption}

\usepackage{amsmath}
\usepackage{listings}

\author{Per R. Leikanger}
                             
\title{Dokumentasjon --Kva eg tenker mens eg koder det.}
\date{\today}     

\begin{document}   

\maketitle

\section{Introduction}
I denne fila skal eg dokumentere kva eg tenker, etterkvart som eg koder. Bra for seinare, når eg skal skrive rapport om det...

%Anngående globale variabler: Skriv i rapporten kvifor eg har brukt referanse-returnerande funksjoner i staden for globale variabler. Referer [Stroustrup, 2008] (3. edition, kap 9) Side 207, der han skriv at det er bra å minimalisere bruk av globale variabler. Referanse-returnerande funk er godt alternativ.
%Sjå s 217.

Kanskje eg skal bruke funksjonsobjekt (for for\_each() ) til bruk på arbeidsliste? Trur det i såfall blir mest for å brife med kunnskap / skrive i rapporten..

Kan også brife med file--streams. Utskrift til matlab-fil for plotting. Dette er også relevant.


\section{Tid}

For å ikkje måtte oppdatere alle objekta i neuralnettet kvar tidsiterasjon, har eg laga konseptet: flytende tid.

Grunnen til å ha tid i det heile tatt er:
\begin{itemize}
	\item at en del aspekter for objekta er tidsvarinte (funksjon av tid). Bl.a. 'refraction time' og lekkasje.
	\item for å få rett kausalitet i det simulerte sytemet. --at  $b_2$ skjer ETTER $b_1$, men samtidig med $a$ om $a$ og $b_2$ skal skje samtidig. Får (simulert) samtidighet.
\end{itemize}

Dette med simulert samtidighet er også eit poeng. Vidare skal vi også få inn 'refraction time' for neurona. %XXX

For å få til dette med tid, uten at alle objekta må skjekkes kva tidsiterasjon, vil eg bruke konseptet flytende tid. For å beskrive dette, må eg først beskrive systemets 'scheduler'.

\subsection{'Scheduler'}
For å få til kausalitet (at $a_2$ skal skje etter $a_1$, men ikkje etter $b$) har eg laga en 'scheduler'. 

Se for deg ei lenka liste som inneholder alle jobbane, pNesteJobbArbeidsKoe. En (alltid nøyaktig en) av jobbane er en spesiell tybe jobb, kalla tid.doTask();. I tillegg har eg en (globalt tilgjengelig) variabel kalla ulTidsiterasjoner.

Alle jobbane har muligheten for å skape nye jobber, som i såfall legges på på slutten av pNesteJobbArbeidsKoe.

Når \emph{tid.doTask()} kalles, vil denne gjøre de aktiviteter som skal gjøres i alle tidsiterasjonane og iterere tid ( ulTidsiterasjoner++ ).
I tillegg skal denne legge på en \emph{[this]} peiker på slutten av arbeidslista \emph{pNesteJobbArbeidsKoe}.

Dette vil skape en lenka liste variant av at det er to arbeidskøer som altererer, med tidsiterator kvar gang ei av de er tom og vi skifter til andre arbeidslista.



\subsection{Klassene}
Alle som skal schedules skal arve fra class tidInterface. Alle klassene som skal ha timing (causalitet og tidsvariante funksjoner) skal arve fra tidInterface. 
Dette er ei interface-klasse med \emph{virtal void doTask() =0;}

I tillegg til alle klassene som har elementer som er funksjoner av tid (for eksempel lekkasjen i 'leaky integrator'), skal en spesialklasse \emph{class tid} arver fra tidInterface.

\subsubsection{tid::pNesteJobbArbeidsKoe}
\emph{tid::pNesteJobbArbeidsKoe} inneholder alle jobber som skal gjøres. Dersom denne er tom, slutter programmet å gjøre ting (i tillegg til at tid slutter --også \emph{tid} er eit element i arbeidslista).

Dersom en jobb skaper nye jobber, legges desse til på slutten av lista.

Ei klasse er heilt spesiell når det kommer til tid: i tillegg til å inneholde \emph{static list pNesteJobbArbeidsKoe}, fungerer tidsSkilleElement som eit skilleElement mellom to tidsiterasjoner. 
Når \emph{tid.doTask()} kalles, skal alt som har med tidsprogresjonen gjøres. I tillegg tenker eg å putte funksjoner som går som bare en funksjon av tid inn her (alt som skal skje i kvar tidsiterasjon)..

Men det viktigaste \emph{tid.doTask()} gjør, er å iterere tid.
\emph{tid} har dermed ansvaret for å skille mellom to tidsiterasjoner.

\subsection{implementasjon}
Som sagt: alle klasser som har ei oppgave som skal gjøres (i den simulerte tida) puttes inn i \emph{pNesteJobbArbeidsKoe}. 

Selve schedulinga skjer i 
\begin{equation}
	\text{void* schedulerFunskjon(void*)}
\end{equation}
Grunnen til formatet på funksjonen er muligheten for å starte tråder (konvensjoner i pthread--library).

Denne funksjonen kaller pNesteJobbArbeidsKoe.front()->doTask(). Når pNesteJobbArbeidsKoe er av type [\emph{class tid}] vil tid itereres.

I tillegg skal alt anna som skal skje kvar einaste iterasjon plasseres her. Eksempel kan være at nye synapser blir lagt i ei kø, og schedulerFunskjon går gjennom denne køa kvar gang den kalles. 
Events kan kanskje også plasseres her?

Eller eg kan gjøre desse andre tinga med egne tråder?



\section{Klassestruktur}
Klassene er bygd opp som i uml-klassediagrammet i loggboka. I tillegg tenker eg å ha interface-klasser for auron og synapse
NEI, har det heller slik at eg har interface--klasse for aktivitetsObj.

Forskjellen mellom SANN og KANN er måten aktiviteten propagerer. Dersom eg da har eit aktivitetsObj med relevante funskjoner (som sendSignal(), getDepol(), osv), så kan synapse, dendrite og auron være det samme for ANN og SANN (bare med ulikt aktivitetsObj (nedarva fra i\_aktivitetsObj, som er interfaceklasse for begge aktivitetsObj--typer).


\include{SANN}
\include{KANN}

\include{KANN-SANN_sammenligning.tex}

\newpage
\section{Dendrite}
Når det kommer til dendritt, tenker eg å holde muligheten open for å utvide. Foreløpig lar eg kvart auron ha bare en dendritt: [\emph{\small{dendrite dendritt\_input;}}].
For bedre simulering kan denne endres til [\emph{\small{std::list$<$dendrite*$>$ alleDentritter;}}].





\section{Synapse}

\subsection{SANN}
Synaptisk vekt skal være unsigned i staden for int!

Ved kvar overføring skal synaptisk vekt legges til eller trekkes fra i postsyn. depol. (avhengig av synapse::bInhibEffekt). 

%Veit ikkje om neste er gyldig lenger. Trur kanskje det. Gidd ikkje sjekke no.
%Først sjekkes det om kor lenge det er siden syn.overføring.  (skjer i synapse::transmission() )
%\begin{lstlisting}
%if( (unsigned uTidSiden = ulTid - ulForrigeOppdatering) != 0) % eller kanskje heller: if(ulTid != ulForrigeOppdatering), og kjøre sammenligninga direkte i pow-funk i gjennomførLekkasje (unngår argumentkopiering..)
%	gjennomførLekkasje(uTidSiden);
%\end{lstlisting}
%
%og gjennomførLekkasje() :
%\begin{lstlisting}
%void gjennomførLekkasje(uTidSiden_arg){
%	depol = lekkasjeFaktor ^ uTidSiden_arg		
%}
%\end{lstlisting}


\subsection{KANN}
KANN--synapsene skal regne ut synapsens innvirking på postsyn. auron. Dette er gitt av presyn. periode og synaptisk vekt. (her kan eg kanskje seinare også legge inn en temp-syn.vekt variabel for å lage short-term syn.p.)

Det er ``synaptisk overføring'' kvar gang presyn. node endrer $\kappa$. 
For at det skal være effektivt i postsyn. dendrite, regner synapsen ut \emph{endring} i synaptisk overføring og sender denne videre. 
For å gjøre dette, trenger synapsen en variabel som holder styr på $\text{[presyn. periode]}^{-1}$, og når vi får synaptisk overføring regner synapsen ut differansen (deriverte) av endinga, og sender dette som argument til dendrite.

\begin{lstlisting}
void synapse::overfoering(){
	unsigned nyPeriode_temp = pMeldemAvAuron->periode; 
	//regnes ut ved A.P. i auronet.
	
	pPostsyn_Dendrite->nyttSignal( synOverfoeringDerivat );  
	//synOverfoeringDerivat er medlemsvariabel (sjaa under)
	
	synOverfoeringDerivat = periode - nyPeriode_temp;
	periode = nyPeriode_temp;
}
\end{lstlisting}

Eller noke tilsvarende.. 

Siden K\_dendrite::nyttSignal( derivat\_arg ) tar derivat som argument, kan vi i dendrite kjøre kallet $K_i += \text{[derivat\_arg]}$. 
For å unngå integral--avvik bør vi kjøre en regelmessig sjekk som rekalkulerer $\kappa_i$ for neuron $i$.
Dette kan gjøres ved å summere alle inn--synapsenes innvirkning (produkt av medlemsvariablane $\text{preSynPeriode]}^{-1}$ og $W_{ij}$.

Kor ofte denne bør kjøres bør enten finnes eksperimentellt, eller gjøres dynamisk (type: dersom integral--avviket er for stort så minskes den regelmessige perioden..).

\newpage





\bibliography{bibliografi}
%\bibliographystyle{abbrvnat}
\bibliographystyle{plain}
\end{document}

